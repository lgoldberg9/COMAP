\documentclass[12pt]{article} 

\usepackage{blindtext}
\usepackage[utf8]{inputenc}
\usepackage[american]{babel}

\usepackage{csquotes}
\usepackage[style=numeric,
    citestyle=numeric,
    backend=biber,
    natbib=true,
url=true]{biblatex}

\usepackage[hidelinks]{hyperref}
\usepackage{setspace}
\usepackage{microtype}

\bibliography{main.bib}

\usepackage[margin=1in]{geometry}

\usepackage{amsmath, amssymb, amsthm}


% Custom macros
\newcommand{\la}{\overleftarrow}
\newcommand{\ra}{\overrightarrow}
\newcommand{\ca}{\overleftrightarrow}
\newcommand{\R}{\mathbb{R}}
\newcommand{\abs}[1]{\left|#1\right|}
\DeclareMathOperator*{\argmin}{\mathrm{arg\,min}}

\title{Numerical simulation of two-dimensional heat convection-diffusion with
ellipsoid geometries}

\date{February 1, 2016}

\begin{document}

\maketitle

\begin{abstract}
	Ever wonder what's the how to keep your warm bath at the right temperature while helping the environment? This paper introduces and analyzes two-dimensional heat convection-diffusion through the Navier-Stokes formulation of fluid-flow and the classical heat equation. Our mission is to determine the optimal amount of water necessary to maintain the perfect bathtub temperature. BLAH BLAH BLAH
\end{abstract}

\section{Introduction}

The basis of this paper is to develop a model that permits a person to maintain their preferred temperature of the bathtub water while minimizing water consumption.    

\section{Model overview}

Throughout this work we will denote the spatial region under scrutiny by
$\Omega \subseteq \R^2$, and we write $\partial \Omega$ to indicate the
boundary of the region. The function $T : \Omega \times \R_{\geq 0} \to \R$
will denote the temperature $T = T(x,y,t)$, and the functions $v_x : \Omega
\times \R_{\geq 0} \to \R$ and $v_y : \Omega \times \R_{\geq 0} \to \R$ will
denote $v_x = v_x(x,y,t)$ the horizontal and $v_y = v_y(x,y,t)$ the vertical
components of the velocity of the fluid flow. 

\subsection{Two-dimensional Navier-Stokes formulation}

The Navier-Stokes equations for an incompressible fluid are given by
\begin{align}
    \nabla \cdot v &= 0 \\
    \frac{\partial v}{\partial t} + (v \cdot \nabla) v &= -
    \frac{1}{\rho} \nabla p + \alpha \nabla^2 v
    \label{eq:\theequation}
\end{align}
where $v = (v_x, v_y)$ is the fluid velocity, $\rho$ denotes mass density of
the fluid, and $p$ denotes pressure. Incompressibility is incorporated
through Equation (), since nonzero divergence implies the presence of either
sources or sinks of fluid. Now because $v$ is two-dimensional, Equation
() actually implies a pair of partial differential equations for both the $x$
and $y$ components. The pressure $p$ is determined through the Poisson equation
\begin{equation}
    \nabla^2 p = b,
    \label{eq:\theequation}
\end{equation}
where $b$ is taken as a constant term for the purposes of this study. Many
wonderful resources exist online for understanding the formulation of these
equations, specifically Barba's work \cite{12-steps}, but we do not include a
derivation of these relations here.

\subsection{Heat convection-diffusion equation}

The heat convection-diffusion equation found in the literature
\cite{convection-diffusion},
\begin{equation}
    \frac{\partial T}{\partial t} - \alpha \nabla^2 T + v \cdot \nabla T =
    0,
    \label{eq:\theequation}
\end{equation}
is a combination of parabolic and hyperbolic partial differential equations
relating the temporal change in temperature with its spatial diffusion and
the flow of heat packets elsewhere in the fluid.     Here, $\nabla$ is the
gradient operator $\left( \frac{\partial}{\partial x},
\frac{\partial}{\partial y} \right)$, $\nabla^2$ denotes the Laplacian
$\frac{\partial^2}{\partial x^2} + \frac{\partial^2}{\partial y^2}$, $v =
(v_x, v_y)$ denotes the fluid velocity vector, and $\alpha$ is a diffusivity
constant.

The ubiquity of parabolic PDEs in modeling scientific phenomenon is well
known, since it captures the essence of diffusive processes. Elliptic PDEs
often arise out of physical environments with a present potential field
(e.g.  gravitational or electrostatic), or in the cases of incompressible
fluid flow \cite{convection-diffusion, ames}. We use the latter for
motivating this model, as bathtub water exhibits the standard
characteristics of incompressible fluid.

\subsection{Numerical methods}

We proceed by providing a brief overview of the numerical methods used in
this study. To approximate the solution to a PDE, we discretize the region
$\Omega$ under consideration into a coordinate mesh. For convention, we
write $T_{i,j}^{n}$ to denote the temperature at the $(i,j)$ mesh coordinate
at the time interval $t=n$. 

Modifying the notation employed by Ames \cite{ames}, we establish the
following nomenclature for the numerical operations in the model:

\begin{center}
    \begin{tabular}[]{ll}
        $\ra{\Delta_i} T_{i,j}^n = T_{i+1,j}^n - T_{i,j}^n$ & Forward
        differencing \\
        $\la{\Delta_i} T_{i,j}^n = T_{i,j}^n - T_{i-1,j}^n$ & Backward differencing \\
        $\ca{\Delta_i} T_{i,j}^n = T_{i+1/2,j}^n - T_{i-1/2,j}^n$ & Central differencing
    \end{tabular}
\end{center}

For 

\subsection{Bathtub optimization problem}

The goal of this study is to identify a process through which the avid bather
may seek to achieve uniform water temperature, ideally wasting as little water
as possible. Using the aforementioned numerical models to simulate bathtub
dynamics, we propose an optimization problem by which to approach this goal. Let
$t_{\mathrm{total}}$ indicate the period in which the faucet is turned on, let
$T_{\mathrm{f}}$ denote the temperature of the water in the faucet, and let
$V(t)$ denote the total volume of water that has left the faucet. We assume that
$\frac{dV}{dt} = c$, i.e., the strength of the faucet is fixed at a chosen $c$ throughout the
duration of the bath. Now, consider the surface integral
\begin{equation}
    f_{\mathrm{cont}}(t) = \int_{\Omega} \abs{T(x,y,t) - T_{\mathrm{f}}} \ d\omega.
    \label{eq:\theequation}
\end{equation}
We propose the minimization problem
\begin{equation}
    \argmin_{c} V(t_{\mathrm{total}}) \quad \text{such that} \quad \int_{0}^{t_{\mathrm{total}}}
    f_{\mathrm{cont}}(t) \ dt < \gamma,
    \label{eq:\theequation}
\end{equation}
where $\gamma$ is some tolerance for variation. This is minimization
problem is subject to a total variation constraint, which has been subject to
considerable study, especially within the context of medical imaging
\cite{Zhang2005}.

This optimization leads to a natural discretization. We rewrite:
\begin{align}
    V(t) &\longrightarrow V^t \\
    \frac{dV}{dt} = c &\longrightarrow V^{t+1} - V^t = c \\
    f_{\mathrm{cont}}(t) = \int_{\Omega} \abs{T(x,y,t) - T_{\mathrm{f}}} \
    d\omega &\longrightarrow f_{\mathrm{disc}}(t) = \sum_{x,y}\abs{T(x,y,t) -
        T_{\mathrm{f}}},
\end{align}
which translate to the problem
\begin{equation}
    \argmin_{c} V^{t_{\mathrm{total}}} \quad \text{such that} \quad
    \sum_{t=0}^{t_{\mathrm{total}}} f_{\mathrm{disc}}(t) < \gamma.
    \label{eq:\theequation}
\end{equation}

\section{Empirical results}

\section{Analysis}

\section{Discussion}

\nocite{air}

\printbibliography

\end{document}
