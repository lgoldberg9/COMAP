\documentclass[10pt]{article} 

\usepackage{blindtext}
\usepackage[utf8]{inputenc}
\usepackage[american]{babel}

\usepackage{csquotes}
\usepackage[style=numeric,
    citestyle=numeric,
    backend=biber,
    natbib=true,
url=true]{biblatex}

\usepackage[hidelinks]{hyperref}
\usepackage{setspace}
\usepackage{microtype}

\bibliography{main.bib}

\usepackage[margin=1in]{geometry}

\usepackage{amsmath, amssymb, amsthm}


% Custom macros
\newcommand{\la}{\overleftarrow}
\newcommand{\ra}{\overrightarrow}
\newcommand{\ca}{\overleftrightarrow}
\newcommand{\R}{\mathbb{R}}
\newcommand{\abs}[1]{\left|#1\right|}
\DeclareMathOperator*{\argmin}{\mathrm{arg\,min}}

\title{The Perfect Bathtub Experience}
\date{February 1, 2016}

\begin{document}

\maketitle

    Ever wonder what's the how to keep your warm bath at the right temperature? We have good news for you! We have discovered an optimal strategy that permits you to maintain the perfect bathtub temperature. We'll begin our discussion of this perfect strategy with how exactly water in the bathtub maintains its warmth.

	Unlike most substances, water is capable of retaining heat for extended periods of time. When a substance heats up, the tiny molecules that compose the substance vibrate more quickly. The vibrating of the external substances molecules is what we classify as temperature. The higher the temperature, the warmer the feel of the substance. 
	
	How does that relate to us humans exactly? Substance molecules interact with one another via electromagnetic forces at a very minute distances. When a substance meets our skin, the tiny molecules push against with our nerve cells using these electromagnetic forces. Subsequently, our nerves respond with feelings of hot, cold, and everything in-between depending on how quickly the substance's molecules are vibrating.
	
	When vibrations in molecules become excessive, the bonds that hold these molecules together will break, releasing or absorbing heat. When a hydrogen bonds to another element, we call this a hydrogen bond; a hydrogen bond, when broken, is an instance where heat is absorbed. Water molecules is made up of two hydrogen bonds. When water heats up, the surrounding heat is absorbed to account for the breaking hydrogen bonds. Consequently, water is able to keep absorb large quantities of heat. 
	
	Now going back to your bathtub, suppose you come home from a long day at work, eager to bathe and unwind. You start to fill your bathtub with warm water from its single faucet. Unfortunately, you do not have a fancy tub with secondary heating systems like circulating jets, so the only way to warm your bathtub is through the faucet. Then you step in to the bathtub to clean and relax. After some time later, you notice the water in the tub is cooler than what you would like. How do you fix this?

	Using the faucet, turn on the water to slightly warmer than you would prefer. Now sure water is great heat retainer, but how about water entering the tub? How can we maintain the specific temperature of the bathtub water? Water is a liquid, so it takes the shape of its container. When water enters the bathtub, it will spread uniformly throughout the bathtub floor and accumulate upon itself due to gravity pulling the water molecules downward. Any substance does not like to stay warm however. Warmer temperatures like to diffuse to lower temperatures to promote a uniform temperature throughout. As such, the water in the bathtub will lose heat to the air above it and the surface of the bathtub. Because the faucet is positioned at one end of the bathtub and your body takes up solid space pushing water to the fringes of the bathtub, it's impossible to have perfectly uniform water temperature in your bath. However, the solution to this problem is to maintain constant water flow in the bathtub, even at a low pressure - this way the warm water will continually cycle into your bathtub and out the drain when it reaches a certain height. The loss of heat to the surrounding environment is taken into account for by the slightly warmer water temperature from the faucet.
	
	Another helpful tactic to maintaining your bathtub temperature is to keep the water constantly in motion. You can help this process by moving around in the bathtub to move the warm water from the faucet to the outer reaches of the bathtub not near the faucet. This tactic helps because warm water from the faucet diffuses its heat to the surrounding cooler water molecules. When colder water molecules are surrounded by warmer water molecules, the bathtub water will heat up faster.
	
	And there it is folks! The perfect bath secrets exposed. To reiterate, the optimal strategy is to maintain a constant input of warm water from the faucet and to assist the warm water flow from to the various sides of the bathtub. These models will put the fears of slowly cooling bathtub behind you and ensure a quality bathing experience.
	
\end{document}
