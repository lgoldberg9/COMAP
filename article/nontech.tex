\documentclass[12pt]{article} 

\usepackage{blindtext}
\usepackage[utf8]{inputenc}
\usepackage[american]{babel}

\usepackage{csquotes}
\usepackage[style=numeric,
    citestyle=numeric,
    backend=biber,
    natbib=true,
url=true]{biblatex}

\usepackage[hidelinks]{hyperref}
\usepackage{setspace}
\usepackage{microtype}

\bibliography{main.bib}

\usepackage[margin=1in]{geometry}

\usepackage{amsmath, amssymb, amsthm}


% Custom macros
\newcommand{\la}{\overleftarrow}
\newcommand{\ra}{\overrightarrow}
\newcommand{\ca}{\overleftrightarrow}
\newcommand{\R}{\mathbb{R}}
\newcommand{\abs}[1]{\left|#1\right|}
\DeclareMathOperator*{\argmin}{\mathrm{arg\,min}}

\title{The Perfect Bathtub Experience}
\date{February 1, 2016}

\begin{document}

\maketitle

    Ever wonder what's the how to keep your warm bath at the right temperature? We have good news for you! We have discovered an optimal strategy that permits you to maintain the perfect bathtub temperature. We'll begin our discussion of this perfect strategy with how exactly water in the bathtub maintains its warmth.

	Unlike most substances, water is capable of retaining heat for extended periods of time. When a substance heats up, the tiny molecules that compose the substance vibrate more quickly. The vibrating of the external substances molecules is what we classify as temperature. The higher the temperature, the warmer the feel of the substance. 
	
	How does that relate to us humans exactly? Substance molecules interact with one another via electromagnetic forces at a very minute distances. When a substance meets our skin, the tiny molecules push against with our nerve cells using these electromagnetic forces. Subsequently, our nerves respond with feelings of hot, cold, and everything in-between depending on how quickly the substance's molecules are vibrating.
	
	When vibrations in molecules become excessive, the bonds that hold these molecules together will break, releasing or absorbing heat. When a hydrogen bonds to another element, we call this a hydrogen bond; a hydrogen bond, when broken, is an instance where heat is absorbed. Water molecules is made up of two hydrogen bonds. When water heats up, the surrounding heat is absorbed to account for the breaking hydrogen bonds. Consequently, water is able to keep absorb large quantities of heat. 
	
	Now going back to your bathtub, suppose you start you come home from a long day at work, eager to bathe and unwind. You start to fill your bathtub with warm water from its single faucet, and 
	
	
\end{document}
